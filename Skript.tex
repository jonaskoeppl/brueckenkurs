\documentclass{report}

% IMPORTS
\usepackage{ntheorem}
\usepackage[ngerman]{babel}
\usepackage[utf8x]{inputenc}
\usepackage{amsfonts}
\usepackage{amsmath}
\usepackage{amssymb}
\usepackage{amstext}
\usepackage{enumerate} % für kleine römische Nummerierung
\usepackage{url}
\usepackage[onehalfspacing]{setspace} %1,5 Zeilenabstand 


% NEWCOMMANDS
% Syntax: \newcommand{\shortcut}{\wasrauskommensoll}

% Shortcuts für Zahlenbereiche:
\newcommand{\N}{\mathbb{N}}
\newcommand{\R}{\mathbb{R}}
\newcommand{\Z}{\mathbb{Z}}
\newcommand{\C}{\mathbb{C}}
\newcommand{\Q}{\mathbb{Q}}
% Stochastik:
\newcommand{\WRaum}{(\Omega, \mathcal{A},\mathbb{P})}
\newcommand{\A}{\mathcal{A}}
\newcommand{\E}{\mathcal{E}}
\newcommand{\Pos}{\mathcal{P}(\Om)}
\newcommand{\Om}{\Omega}
\newcommand{\s}{\sigma}
\newcommand{\Bor}{\mathcal{B}}
\newcommand{\mbr}[1]{$(\Om_{#1}, \A_{#1})$}
\newcommand{\topspace}[1]{$(\Om_{#1}, \mathcal{O}_{#1})$}
\newcommand{\Fbar}{\bar{\mathcal{F}}(\Om, \A)}
\newcommand{\Fbarplus}{\bar{\mathcal{F}}_+(\Om, \A)}
\newcommand{\fmb}{\mathcal{F}(\Om, \A)}


% THEOREMSTYLES

\theorembodyfont{\upshape}
\theoremstyle{changebreak} %Sorgt für Zeilenumbruch nach Titel und Nummer vor Name

\newtheorem{theorem}{Satz} % Sätze, sollte später evtl. noch Lemma und Korollar hinzufügen
\numberwithin{theorem}{chapter}

\newtheorem{mydef}[theorem]{Definition} % Definitionen
\numberwithin{theorem}{chapter}

\newtheorem{example}[theorem]{Beispiel} % Beispiele
\numberwithin{theorem}{chapter}

\newtheorem{proposition}[theorem]{Proposition} % Propositionen
\numberwithin{theorem}{chapter}

\newtheorem{remark}[theorem]{Bemerkung} %Bemerkungen
\numberwithin{theorem}{chapter}

\newtheorem{lemma}[theorem]{Lemma} %Lemmata
\numberwithin{theorem}{chapter}

\newtheorem*{proof*}{Beweis:}

\setlength\parindent{0pt} %Kein Einzug bei neuem Paragraphen. 

\begin{document}

\begin{titlepage}
    \ \newline\newline\newline\newline\newline

	\begin{center}

		\huge Skriptum \\
		\Huge\textbf{Brückenkurs Mathematik}\\
		\huge im Wintersemester 2018/2019\\ 
		\normalsize

		\vspace{1cm}
		\begin{tabular}[b]{l|l}
			\textbf{\ author}      & Jonas Köppl, N.N. \\\
            \textbf{last change}   & \today \\\
            \textbf{github}        & \url{TODO}
		\end{tabular}
		\vspace{1cm}

	\end{center}

\end{titlepage}

\tableofcontents 


\chapter{Einführung in das mathematische Arbeiten}
% Subsection: 1.1. Ziel und Inhalt
\section{Ziele und Inhalt}
Im Vergleich mit vielen anderen Studiengängen, selbst mit den naturwissenschaftlichen,
hat das Mathematikstudium eine höhere Abbruchrate, und viele Studenten geben bereits während
oder nach dem ersten Semester auf. Ein Hauptgrund für diese Begebenheit/Tatsache ist, dass
sich die Art und Weise wie Mathematik an der Universität betrieben wird, grundlegend von dem
unterscheidet, was man aus der Schule gewohnt ist. [...] Ziel des Brückenkurses ist es also die
Studierenden auf eine schonende/sanfte Art auf ihr erstes Semester vorzubereiten und sie schoneinmal
mit der Hochschulmathematik vertraut zu machen. Zu den Inhalten zählen neben einer Einführung
in die Aussagenlogik auch die naive Mengenlehre und elementare Zahlentheorie. Diese bilden
eine gute Grundlage um anhand von Beispielen verschiedene Beweistechniken kennzulernen und
mit Übungsaufgaben zu vertiefen.

% Subsection 1.2. TODO: Schwierigkeiten zu Studienbeginn
\section{Schwierigkeiten beim Studienbeginn}
Auch aus eigener Erfahrung wissen wir, dass die folgenden beiden Punkte zu Problemen im ersten
Semester führen:
\begin{enumerate}
    \item
    Die scheinbare Einfachheit des zu Beginn gelehrten Stoffes(schwer den Punkt zu finden an
    dem man anfangen sollte zu lernen, man lässt zu früh abreißen).
    \item
    Der hohe Abstraktionsgrad, mit Beispielen kommt man nicht mehr weit, man weiß nicht auf
    was man achten soll, abstrakte Strukturen schocken einen. (-> metrische Räume zur Einführung).
\end{enumerate}
Also: Nicht abreißen lassen, Übungsaufgaben machen, Fragen stellen, Lerngruppe suchen.

Zu Beginn / Zum Aufwärmen: Einführende Rätsel, bisschen warm werden, mit den Sitznachbarn sprechen.

\include{chapter/01-Grundlagen}
\include{chapter/02-Mengen}
%TODO: Rechenregeln für Mengenoperationen, DeMorgan und Indexmengen
\chapter{Aufbau mathematischer Theorien}

In diesem Abschnitt sprechen wir über den allgemeinen Aufbau einer mathematischen Theorie bzw. eines mathematischen Textes. 
Anhand zahlentheoretischer Beispiele werden zudem erste Beweistechniken eingeübt. 

\section{Definitionen}

Jeder von euch kennt bereits eine Vielzahl von Definitionen. Nicht nur aus den vorherigen Kapiteln sondern auch aus der Schule. 
Was ist ein Dreieck, Parallelogramm oder Quadrat? Was meint man mit rechtwinklig? Was ist ein größter gemeinsamer Teiler?
Ziel einer mathematischen Definition ist es, einen Begriff durch \textit{invariante} Merkmale so zu beschreiben, dass man für jedes 
vorliegende mathematische Objekt \textit{eindeutig} entscheiden kann, ob dieses Objekt der Definition der genügt oder nicht. 

\begin{mydef}
    Seien $a,b \in \Z$. Die Zahl $a$ heißt \textit{Teiler} von $b$ und $b$ heißt \textit{Vielfaches} von $a$, wenn es ein $c \in \Z$ gibt mit 
    $b = a \cdot c$. Man schreibt dann $a \mid b$. Die Negation ist $a \nmid b$. 
\end{mydef}

\begin{mydef}
    Sei $n \in \Z$. 
    \begin{enumerate}
        \item 
        Die Zahl $n$ heißt \textit{gerade}, falls $2 \mid n$. 
        \item 
        Falls $n$ nicht gerade ist, so heißt $n$ \textit{ungerade}. 
    \end{enumerate}
\end{mydef}

\begin{mydef}
    Eine Zahl $p \in \N$ heißt \textit{Primzahl}, falls $p > 1$ ist und $1$ und $p$ die einzigen natürlichen Zahlen sind die $p$ teilen. Eine Zahl $n \in \N$ die keine Primzahl ist heißt \textit{zusammengesetzt}. 
\end{mydef}

Eine mathematische Definition sollte folgende Eigenschaften haben: 
\begin{itemize}
    \item 
    \textbf{Eindeutigkeit}: Man kann eindeutig entscheiden, ob ein mathematische Objekt die Definition erfüllt oder nicht. 
    \item 
    \textbf{Minimalität}: Es werden nur so viele Eigenschaften benutzt, wie für den Begriff unbedingt notwendig sind. 
    D.h. insbesondere, dass sich keine der Eigenschaften aus den restlichen Eigenschaften ableiten lässt. 
\end{itemize}

\section{Sätze und Beweise}

Unter einem mathematischen Satz (Lemma, Korollar, etc.) verstehen wir eine nicht-triviale mathematische Aussage, für die ein gültiger Beweis vorliegt.

\begin{remark}
    Zur Begriffsklärung: 
    \begin{enumerate}[(i)]
        \item
        \textbf{Satz, Theorem}: Dies ist das typische Resultat einer Theorie.
        \item
        \textbf{Hauptsatz}: So wird ein besonders wichtiger Satz in einem Teilgebiet der Mathematik
        genannt, beispielsweise der Hauptsatz der Differential- und Integralrechnung aus der Analysis.
        \item
        \textbf{Lemma}:
        Diese Bezeichnung wird in verschiedenen Zusammenhängen verwendet. Zum einen bezeichnet es
        ein kleines, meist technisches Resultat, einen \textit{Hilfssatz}, der zum Beweis eines wichtigen
        Satzes verwendet wird. Zum anderen handelt es sich dabei um besonders wichtige \textit{Schlüsselgedanken},
        die in vielen Situationen nützlich sind. Solche Lemmata werden dann auch häufig mit dem Namen ihres Erfinders bezeichnet
        (z.B. Lemma von Pratt, Lemma von Urysohn, Lemma von Zorn, etc.).
        \item
        \textbf{Proposition}:
        Dies ist die lateinische Bezeichnung für Satz und wir manchmal synonmym verwendet,
        meist aber um ein Resultat zu bezeichnen, dessen Wichtigkeit zwischen der eines Hilfssatzes
        und der eines Theorems liegt.
        \item
        \textbf{Korollar}: 
        Dies ist die Bezeichnung für einen Satz, der aus einem anderen Satz durch triviale oder sehr einfache
        Schlussweise folgt. Manchmal ist es aber auch ein Spezialfall eines vorhergehenden Satzes, dem besondere Aufmerksamkeit gebührt.
    \end{enumerate}

\end{remark}

\subsection{Direkter Beweis}
Der direkte Beweis beweist eine Aussage durch schrittweises Folgen von Aussagen auf Basis der gegebenen Voraussetzung. 
Sei dazu $V$ die Voraussetzung und $B$ die zu zeigende Behauptung. Unsere Aufgabe ist es also nun, geeignete Aussagen $A_1, A_2,...,A_n$ zu finden, 
so dass schließlichgilt: 
\begin{align*}
    (V \Rightarrow A_1) \wedge (A_1 \Rightarrow A_2) \wedge ... \wedge (A_n \Rightarrow B)
\end{align*}
Ist uns dies gelungen, so haben wir gezeigt, dass aus der Voraussetzung $V$ stets auch die Behauptung $B$ folgt. 
Am Besten veranschaulichen wir uns dies anhand eines einfachen Beispiels aus der Zahlentheorie. 
\begin{proposition}
    Sei $a \in \Z$ gerade. Dann ist auch $a^2 = a \cdot a$ gerade. 
\end{proposition}


\begin{proof*}
    Sei $a \in \Z$ gerade. Nach Definition existiert dann $c \in \Z$ mit $a = 2 \cdot c$. Somit folgt: 
    \begin{align*}
        a^2 = a \cdot a = 2c \cdot 2c = 2(2c^2).
    \end{align*}
    Wegen $2c^2 \in \Z$ ist also auch $a^2$ gerade. 
    \hfill $\square$
\end{proof*}

Allgemeiner lassen sich die folgenden Regeln zeigen:

\begin{theorem}
    \begin{enumerate}[(i)]
        \item 
        Für jedes $a \in \Z$ gilt: 
        \begin{align*}
            a \mid 0, 1 \mid a, -1 \mid a, a \mid a, -a \mid a, a \mid -a
        \end{align*}
        \item 
        Für $a,b,c \in \Z$ gilt: 
        \begin{align*}
            (a \mid b \ \wedge \ b \mid c) \Rightarrow a \mid c 
        \end{align*}
        \item 
        Seien $a,b_1,...,b_n \in \Z$. Gilt $a \mid b_i$ für alle $i \in \{1,...,n\}$, so gilt für alle $x_1,...,x_n \in \Z$: 
        \begin{align*}
            a \mid x_1 b_1 + ... + x_n b_n 
        \end{align*}
        \item 
        Für $a,b,c,d \in \Z$ gilt: 
        \begin{align*}
            (a \mid c \ \wedge  \ b \mid d) \Rightarrow ab \mid dc 
        \end{align*}
    \end{enumerate}
\end{theorem}

\begin{proof*}
    [Zur Übung]
\end{proof*}

\subsection{Indirekter Beweis}
Eine weitere oft verwendete Beweistechnik ist die des \textit{Indirekten Beweis}. Diese beruht auf der folgenden logischen Äquivalenz: 
Seien $V$ und $B$ zwei Aussagen, dann gilt: 

\begin{align}
    (V \Rightarrow B) \iff (\neg B \Rightarrow \neg V)
\end{align}

Diese Beweistechnik bietet sich in einigen Fällen an, da die rechte Implikation manchmal einfacher zu zeigen ist als die linke. 
Zusammengefasst ergibt sich das folgende Schema um eine Aussage der Form $A \Rightarrow B$ zu zeigen: 
\begin{enumerate}
    \item 
    Wir nehmen an es gelte $\neg B$ (und bringen dies auch zum Ausdruck, sodass auch der Leser oder Korrektor sieht was wir vorhaben).
    \item 
    Aus der Aussage $\neg B$ und anderen uns zur Verfügung stehenden Definitionen und Sätzen leiten wir $\neg V$ ab. 
    \item 
    Wegen der oben beschrieben Äquivalenz gilt dann auch $V \Rightarrow B$. 
\end{enumerate}
Diese Vorgehensweise werden wir nun verwenden um den folgenden Satz zu beweisen. 
\begin{theorem}
    Sei $n \in \Z$, sodass $n^2$ gerade ist. Dann ist auch $n$ gerade. 
\end{theorem}
\begin{proof*}
    Sei also $n \in \Z$ ungerade. Wir zeigen nun, dass dann auch das Quadrat von $n$ ungerade ist. Da $n$ ungerade ist existiert $c \in \Z$ mit $n = 2c + 1$. 
    Somit erhalten wir durch Anwendung der ersten binomischen Formel: 
    \begin{align*}
        n^2 = (2c +1)^2 = 4c^2 + 4c + 1 = 2(2c^2+2c) + 1 
    \end{align*}
    Also ist $n^2$ ungerade und die Behauptung gezeigt. 
    \hfill $\square$
\end{proof*}

Für diese Beweistechnik ist es sehr wichtig, dass man sich darüber im Klaren ist, wie die Negation einer Aussage lautet. Daher wollen wir dies mit ein paar Beispielen vertiefen. 
\begin{example} % TODO: Übungsaufgabe, ggf auf Übungsblatt packen. 
    Formulieren Sie die logische Negation von: 
    \begin{align*}
        \forall x \in \R: \forall \varepsilon > 0 \ \exists y \in \Q : -\varepsilon < x - y < \varepsilon
    \end{align*}
\end{example}

\subsection{Widerspruchsbeweis} %TODO: überarbeiten. 

Der \textit{Widerspruchsbeweis} (auch bekannt als \textit{Reductio ad absurdum}) basiert auf der logischen Äquivalenz: 
\begin{align*}
    (V \Rightarrow B) \iff \neg (V \wedge \neg B)
\end{align*}

Ein Beweis per Widerspruch verläuft also nach dem folgenden Schema: 
\begin{enumerate}
    \item 
    Wir bringen zum Ausdruck, dass der Beweis per Widerspruch erfolgen soll. Meist schreibt man einfach: "Widerspruchsannahme: $\neg B$". 
    Auch hier ist es wichtig, dass man die Negation der Aussage $B$ richtig formuliert. 
    \item 
    Aus den Aussagen $V$ und $\neg B$ leiten wir nun eine Aussage ab, von der wir bereits wissen, dass sie falsch ist. 
    \item 
    Um zu zeigen, dass dies der gewünschte Widerspruch ist markieren wir die Stelle durch einen Blitz oder durch das Wort "Widerspruch". 
\end{enumerate}

Da diese Beweistechnik einem nicht direkt einleuchtend erscheint werden wir uns ein paar Beispiele dazu anschauen. 

\begin{lemma}
    \begin{enumerate}[(i)]
        \item 
        Ist $b \in \Z \setminus \{0\}$ so gilt für jeden Teiler $a$ von $b$ : 1 $\leq \lvert a \rvert \leq \lvert b \rvert$. 
        \item 
        Die einzigen Teiler von $1$ sind $1$ und $-1$. 
        \item 
        Für $a,b \in \Z$ gilt: 
        \begin{align*}
            a \mid b \ \wedge \ b \mid a \iff b = a \text{  oder } b = -a
        \end{align*}
    \end{enumerate}
\end{lemma}

\begin{proof*}
    \textbf{zu (i):} \newline
    Seien $a,b \in \Z$ mit $b \neq 0$ und gelte $a \mid b$. Dann existiert ein $n \in \Z$ mit $b = n \cdot a$ und somit gilt auch $n \neq 0$. Die erste Ungleichung ist wegen $b \neq 0$ klar. Angenommen es gilt $\lvert a \rvert > \lvert b \rvert$. 
    Dann folgt unter Verwendung elementarer Rechenregeln für die Betragsfunktion: 
    \begin{align*}
        \lvert b \rvert = \lvert n a \rvert = \lvert n \rvert \lvert a \rvert \geq \lvert a \rvert > \lvert b \rvert
    \end{align*}
    Widerspruch. Also gilt $1 \leq \lvert a \rvert \leq \lvert b \rvert$.
    \newline 
    \textbf{zu (ii),(iii):} [Zur Übung]. \hfill $\square$
\end{proof*}

\begin{lemma}
    Sei $a \in \N$ mit $a > 1$. Dann gibt es $r \in \N$ und Primzahlen $p_1,...,p_r$, sodass: 
    \begin{align}
        a = p_1 \cdot ... \cdot p_r
    \end{align}
    Die Zerlegung (4.2) wird auch als die \textit{Primfaktorzerlegung} von $a$ bezeichnet. 
\end{lemma}

\begin{proof*}%TODO: überarbeiten
    Angenommen die Behauptung des Lemmas ist falsch. Dann gibt es insbesondere eine kleinste natürliche Zahl $a$ mit $a>1$ für die keine derartige Zerlegung existiert. 
    Zunächst einmal halten wir fest, dass dann $a$ keine Primzahl sein kann. Denn sonst wäre $a$ ja trivialerweise ein Produkt von Primzahlen. 
    Also gibt es eine Zahl $b \in \N \setminus \{1,a\}$ mit $b \mid a$. Daher existiert nach Definition $c \in \N$ mit $a = b \cdot c$. Nach Satz XXXX gilt ferner: 
    $b < a$ und $c < a$. Da $a$ die kleinste natürliche Zahl ist, die keine derartige Zerlegung besitzt lassen sich $b$ und $c$ in Primfaktoren zerlegen: 
    \begin{align*}
        b &= p_1 \cdot ... \cdot p_s \\\
        c &= p_{s+1} \cdot ... \cdot p_k
    \end{align*}
    Somit folgt: 
    \begin{align*}
        a = b c = p_1 ... p_s p_{s+1} ... p_k 
    \end{align*}
    Im Widerspruch zur Voraussetzung. Also besitzt jede natürliche Zahl eine Primfaktorzerlegung. \hfill $\square$
\end{proof*}

Auf Basis dieses Lemmas können wir nun den folgenden, auf Euklid zurückgehenden, Satz beweisen. Auch diesen werden wir per Widerspruch beweisen. 

\begin{theorem}
    Es gibt unendlich viele Primzahlen. 
\end{theorem}

\begin{proof*}
    Die Negation von "unendlich viele" ist "endlich viele". Also nehmen wir an, dass es nur endlich viele Primzahlen gibt. Sei also 
    \begin{align*}
        P = \{p_1,...,p_n\}
    \end{align*}
    die Menge aller Primzahlen. Setze nun
    \begin{align*}
        a = p_1 ... p_n + 1 
    \end{align*}
    Dann gilt offensichtlich $a>1$. Nach Lemma xx.xx lässt sich $a$ also in Primfaktoren zerlegen. Es gilt also $p_i \mid a$ für ein $i \in \{1,...,n\}$. 
    Wegen $p_i \mid p_1 ... p_n$ gilt nach Satz xx.xx : $p_i \mid a - p_1 ... p_n = 1$. Also $p_i \mid 1$. Dies liefert $p_i = 1$. Im Widerspruch dazu, dass $p_i$ eine Primzahl ist.\hfill $\square$ 
\end{proof*}

Ein weiteres klassisches Resultat welches man per Widerspruch beweist ist der folgenden Satz, den wir schon zuvor als Beispiel gesehen haben. 
\begin{theorem}
    Die Zahl $\sqrt{2}$ ist irrational. D.h. ist $q \in \Q$, so gilt $q \neq \sqrt{2}$. 
\end{theorem}
\begin{proof*}
    Angenommen $\sqrt{2}$ ist rational. Dann existieren $p,q \in \N$ mit $\sqrt{2} = \frac{p}{q}$. Ferner seien $p$ und $q$ teilerfremd. 
    Es gilt also insbesondere: 
    \begin{align*}
        (\frac{p}{q})^2 = 2 . 
    \end{align*}
    Somit folgt durch umstellen: 
    \begin{align*}
        p^2 = 2q^2.
    \end{align*}
    Also ist $p^2$ eine gerade Zahl. Nach Satz xx.xx ist also auch $p$ gerade, folglich existiert ein $n \in \N$ mit $p = 2n$. 
    Dies liefert wiederum: 
    \begin{align*}
        2q^2 = p^2 = (2n)^2 = 4n^2
    \end{align*}
    Also gilt insbesondere $q^2 = 2n^2$. Also ist auch $q$ durch zwei teilbar. Im Widerspruch dazu, dass $p$ und $q$ teilerfremd sind. 
    Somit kann $\sqrt{2}$ keine rationale Zahl sein und die Behauptung ist gezeigt. \hfill $\square$. 

\end{proof*}

\subsection{Allgemeine Tipps für Beweise}

Für nahezu jeden Studienanfänger ist es eine der größten Hürden in den ersten Wochen und Monaten des Studiums, seine eigenen Gedanken sinnvoll und nachvollziehbar zu Papier zu bringen.
Dies führt bei vielen zu Frustration, da die Gedankengänge an sich oftmals in die richtige Richtung gehen, aber nicht in mathematisch korrekter Weise formuliert werden, was meist zu Punktabzügen führt. 
Im Folgenden geben wir einige Tipps, die beim Aufschreiben eines Beweises bzw. beim Bearbeiten von Übungsaufgaben helfen können. 
\newline 
\newline 
\textbf{Voraussetzungen(Setting)}:\newline 
Der erste Schritt eines jeden Beweises ist es die Voraussetzungen klar und deutlich aufzuschreiben. 
Es ist oftmals sinnvoll die einzelnen Teilaussagen des Settings durchzunummerieren. 
\newline\newline  
\textbf{Skizze}:\newline 
Manchmal kann es durchaus hilfreich sein, sich eine Skizze von einem Sachverhalt zu machen. Dies dient dazu, ein besseres Verständnis des Problems zu erhalten. 
Ist das Problem geometrischer Natur so kann man sich auch die Beweisidee in die Skizze einzeichnen. Dies dient jedoch lediglich zur Veranschaulichung und ist auf keinen Fall ein Beweis. 
\newline\newline 
\textbf{Fallunterscheidung}:\newline
Bevor mit dem eigentlichen Beweis begonnen wird kann man sich fragen, ob man das Problem in zwei Teilprobleme aufteilen sollte, die beide einfacher wären, als das große Problem auf einmal zu lösen. 
Auch Spezialfälle können per Fallunterscheidung ausgegliedert werden. 
\newline\newline 
\textbf{OBdA / OE}:\newline
OBdA ist die Abkürzung von \textit{Ohne Beschränkung der Allgemeintheit} und OE steht für \textit{Ohne Einschränkung}. Dies wird verwendet, wenn eigentlich eine Fallunterscheidung angewendet werden müsste, jedoch alle anderen Fälle einfach aus dem behandelten Fall gefolgert werden können. 
Dieses Mittel sollte auf jeden Fall mit Bedacht eingesetzt werden. 
\newline\newline 
\textbf{Gegenbeispiele}:\newline
Gegenbeispiele sind das Mittel der Wahl um eine falsche Aussage zu widerlegen. 
\newline\newline 

\textbf{Verwendung von Symbolen}:\newline
Symbole im Text erhöhen zwar oft dessen Präzision und machen den Text an sich kürzer, allerdings sollte man zur besseren Lesbarkeit ein paar Regeln beachten.
\begin{enumerate}
    \item 
    Ein Satz sollte nicht mit einem Symbol beginnen. 
    \item
    Zwei mathematische Symbole sollten im Fließtext stets durch mindestens ein Wort getrennt werden. 
    \item 
    Mathematische Symbole sollten nicht als Abkürzung für Worte im Text verwendet werden. 
\end{enumerate}\
\newline
\textbf{Lesbarkeit und Übersichtlichkeit}: \newline 
Ein einfacher aber sehr effektiver Weg einen Beweis zu verbessern besteht darin euren Beweis übersichtlich zu strukturieren und vor allem leserlich zu schreiben. 
Insbesondere kann der Korrektor so euren Gedankengang deutlich besser nachvollziehen, wodurch ihr ein besseres Feedback erhaltet.







%TODO: Widerspruchsbeweis überarbeiten und Teilbarkeitsbeweise genauer anschauen. 
\chapter{Relationen und Abbildungen}

Ziel dieses Kapitel ist es, den Begriff der Abbildung (oder Funktion) mathematisch präzise zu fassen. 
Da eine Funktion ein Spezialfall einer Relation ist werden wir uns zuerst einmal mit Relationen im Allgemeinen befassen.

\section{Relationen}

\begin{mydef}
    Seien $A,B$ zwei nichtleere Mengen. Dann ist das \textit{kartesische Produkt} von $A$ und $B$ definiert durch:
    \begin{align*}
        A \times B := \{ (a,b) | a \in A \wedge b \in B \}. 
    \end{align*}
\end{mydef}

\begin{mydef}
    Seien $A,B$ zwei nichtleere Mengen und $A \times B$ das kartesische Produkt von $A$ und $B$. 
    Eine Teilmenge $R \subseteq A \times B$ heißt dann auch eine (zweistellige) \textit{Relation} (zwischen $X$ und $Y$).
    Gilt $X = Y$ so spricht man auch von einer Relation \textit{auf} $X$.  
    Anstatt von $(x,y) \in R$ schreibt man auch $xRy$. 
\end{mydef}

\begin{example}
    Die Folgenden Mengen sind Relationen: 
    \begin{enumerate}[(i)]
        \item 
         $R_1 := \{(a,b) | a \neq b \} \subseteq \Z \times \Z$.   
        \item 
        $R_2 := \{ (a,b) | a | b \} \subseteq \Z \times \Z$.
        \item 
        $R_3 := \{(a,b) | a \leq b\} \subseteq \R \times \R$. 
    \end{enumerate}
\end{example}

Eine Relation kann verschiedene Eigenschaften haben: 

\begin{mydef} 
    Seien $X$ eine nichtleere Meng und $R \subseteq X \times X$ eine Relation. $R$ heißt
    \begin{enumerate}[(i)]
        \item 
        \textit{reflexiv}, falls: $\forall x \in X: (x,x) \in R$,
        \item 
        \textit{symmetrisch}, falls: $\forall x,y \in X : (x,y) \in R \Rightarrow (y,x) \in R$,
        \item 
        \textit{antisymmetrisch}, falls: $\forall x,y \in X: (x,y) \in R \wedge (y,x) \in R \Rightarrow x=y$,
        \item 
        \textit{transitiv}, falls: $\forall x,y,z \in X: (x,y) \in R \wedge (y,z) \in R \Rightarrow (x,z) \in R$. 
    \end{enumerate}
\end{mydef}

TODO: Hierzu ein Beispiel. 

\begin{mydef}
    Eine Relation $R$ auf einer Menge $X$ wird als \textit{Äquivalenzrelation} bezeichnet, falls $R$ reflexiv, symmetrisch und transitiv ist. 
\end{mydef}

\begin{mydef}
    Eine Relation $R$ auf einer Menge $X$ wird als \textit{partielle Ordnung} bezeichnet, falls $R$ reflexiv, antisymmetrisch und transitiv ist. 
\end{mydef}

\begin{proposition}
    Sei $X$ eine Menge. Dann ist durch 
    \begin{align*}
        R := \{ (A,B) | A \subseteq B \}
    \end{align*}
    eine partielle Ordnung auf $\mathcal{P}(X)$ definiert. 
\end{proposition}

Beweis zur Übung? 

\begin{proposition}
    Sei $m \in \Z$. Dann ist durch 
    \begin{align*}
        R_m := \{(a,b)| m \mid (b-a)\}
    \end{align*}
    eine Äquivalenzrelation auf $\Z$ definiert. 
\end{proposition}

Beweis zur Übung? 

\section{Abbildungen}

\begin{mydef}
    Seien $A,B$  nichtleere Mengen und $R$ eine Relation zwischen $A$ und $B$. $R$ heißt 
    \begin{enumerate}[(i)]
        \item 
        \textit{linkstotal}, falls: $\forall a \in A \ \exists b \in B : (a,b) \in R$. 
        \item
        \textit{rechtseindeutig}, falls: $\forall a \in A \forall b,c \in B: (a,b) \in R \wedge (a,c) \in R \Rightarrow b = c $. 
    \end{enumerate}
    Eine linkstotale und rechtseindeutige Relation wird als \textit{Abbildung} oder auch als \textit{Funktion} bezeichnet. 
\end{mydef}

\begin{remark}
    Wenn man die Begriffe der Linktstotalität und der Rechtseindeutigkeit zusammenfasst erhält man 
    \begin{align}
        \forall a \in A \exists ! b \in B : (a,b) \in R. 
    \end{align}
    Dies wird in der Literatur vereinzelt auch als \textit{Funktionseigenschaft} bezeichnet. 
\end{remark}

\begin{remark}
    Funktionen können natürlich mit der für Relationen kennengelernten Notation aufgeschrieben werden, allerdings hat sich in der heutigen Zeit die folgende Notation
    für eine Funktion $f$ von $A$ nach $B$ durchgesetzt: 
    \begin{align*}
        f : A \to B, x \mapsto f(x)
    \end{align*}
    Hierbei ist es wichtig anzumerken, dass jede Funktionsdefinition aus zwei Bausteinen besteht: 
    \begin{enumerate}
        \item Angabe von Definitions- und Wertebereich,
        \item Angabe der Abbildungsvorschrift. 
    \end{enumerate}
\end{remark}

\begin{example}
    \begin{enumerate}[(i)]
        \item
        $f: \Z \to \Z, x \mapsto x^2 $. 
        \item 
        Sei $X$ eine nichtleere Menge. Dann heißt die Abbildung 
        \begin{align*}
            id_X : X \to X, x \mapsto x
        \end{align*}
        die \textit{Identität} auf $X$. 
        \item 
        Nur durch $x \mapsto x^2$ ist keine Funktion definiert, da Definitions- und Wertebereich nicht spezifiziert sind. 
    \end{enumerate}
\end{example}

Oftmals ist es notwendig Funktionen auf spezielle Eigenschaften zu untersuchen. Einige dieser Eigenschaften,
die euch im Laufe des Studiums noch häufiger begegnen werden, werden im Folgenden vorgestellt. %TODO: Satzbau. 

\begin{mydef}
    Seien $A,B$ nichtleere Mengen. Eine Abbildung $f:A \to B$ heißt
    \begin{enumerate}[(i)]
        \item
        \textit{injektiv}, falls: 
        \begin{align*}
            \forall a_1,a_2 \in A: f(a_1) = f(a_2) \Rightarrow a_1 = a_2, 
        \end{align*}
        \item 
        \textit{surjektiv}, falls: 
        \begin{align*}
            \forall b \in B \exists a \in A: f(a) = b,
        \end{align*}
        \item 
        \textit{bijektiv}, falls $f$ surjektiv und injektiv ist. 
    \end{enumerate}
\end{mydef}

\begin{example}
    Betrachte die Funktion 
    \begin{align*}
        f: \Z \to \Z, \ x \mapsto x^2.
    \end{align*}
    Dann ist $f$ nicht injektiv, denn es gilt: $f(-5) = 25 = f(5)$ und $-5 \neq 5$. Zudem ist $f$ auch nicht surjektiv, denn es gilt $f(x) = z^2 \geq 0$ für alle $x \in \Z$. 
\end{example}

\section{Bild und Urbild} 

\begin{mydef}
    Seien $X,Y$ nichtleere Mengen und sei $f:X \to Y$ eine Abbildung. 
    \begin{enumerate}[(i)]
        \item 
        Sei $A \subseteq X$. Dann heißt $f(A):=\{y \in Y | \exists x \in X : f(x) = y \}$ das \textit{Bild} von $A$ unter $f$. 
        \item 
        Sei $B \subseteq Y$. Dann heißt $f^{-1}(B):=\{x \in X | f(x) \in B \}$ das \textit{Urbild} von $B$ unter $f$. 
    \end{enumerate}
\end{mydef}

\begin{example}
    Betrachte die Abbildung 
    \begin{align*}
        f: \R to \R, x \mapsto -x,
    \end{align*}
    Sowie die Mengen $A = [0,2]$ und $B = \{1,3,8,9\}$. Dann gilt: 
    \begin{align*}
        f(A) &= [-2,0], \\\
        f^{-1}(A) &= \{-1, -3,-8,-9\}. 
    \end{align*}
\end{example}

\begin{theorem}
    Seien $X,Y$ nichtleere Mengen, $f:X \to Y$ eine Abbildung, $I$ eine Menge, seien $A,B$ und $A_i$, $i \in I$, Teilmengen von $X$. Dann gilt: 
    \begin{enumerate}
        \item
        $f(A) = \emptyset \iff A = \emptyset$, 
        \item 
        $A \subseteq B \Rightarrow f(A) \subseteq f(B)$,
        \item 
        $ f(\bigcup_{i \in I}A_i) = \bigcup_{i \in I}f(A_i)$,
        \item 
        $ f(\bigcap_{i \in I}A_i) \subseteq \bigcap_{i \in I}f(A_i)$, wobei im Allgemeinen keine Gleichheit gilt. 
    \end{enumerate}
\end{theorem}

\begin{proof*}
    [Zur Übung]
\end{proof*}

\begin{theorem}
    Seien $X,Y$ nichtleere Mengen, $f:X \to Y$ eine Abbildung, $I$ eine Menge, seien $A,B$ und $A_i$, $i \in I$, Teilmengen von $Y$. Dann gilt: 
    \begin{enumerate}[(i)]
        \item 
        $f^{-1}(A) = \emptyset \iff A \cap f(X) = \emptyset$,
        \item 
        $A \subseteq B \Rightarrow f^{-1}(A) \subseteq f^{-1}(B)$,
        \item 
        $f^{-1}(\bigcup_{i \in I}A_i) = \bigcup_{i \in I}f^{-1}(A_i)$,
        \item 
        $f^{-1}(\bigcap_{i \in I}A_i) = \bigcap_{i \in I}f^{-1}(A_i)$,
        \item 
        $f^{-1}(Y \setminus A) = X \setminus f^{-1}(A)$.
    \end{enumerate}
\end{theorem}

\begin{proof*}
    [Zur Übung]
\end{proof*}


\end{document}
