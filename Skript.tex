\documentclass{report}

% IMPORTS
\usepackage{ntheorem}
\usepackage[ngerman]{babel}
\usepackage[utf8x]{inputenc}
\usepackage{amsfonts}
\usepackage{amsmath}
\usepackage{amssymb}
\usepackage{amstext}
\usepackage{enumerate} % für kleine römische Nummerierung
\usepackage[onehalfspacing]{setspace} % 1,5 Zeilenabstand
\usepackage[hyphens]{url}
\usepackage{hyperref}

% NEWCOMMANDS
% Syntax: \newcommand{\shortcut}{\wasrauskommensoll}

% Shortcuts für Zahlenbereiche:
\newcommand{\N}{\mathbb{N}}
\newcommand{\R}{\mathbb{R}}
\newcommand{\Z}{\mathbb{Z}}
\newcommand{\C}{\mathbb{C}}
\newcommand{\Q}{\mathbb{Q}}
% Stochastik:
\newcommand{\WRaum}{(\Omega, \mathcal{A},\mathbb{P})}
\newcommand{\A}{\mathcal{A}}
\newcommand{\E}{\mathcal{E}}
\newcommand{\Pos}{\mathcal{P}(\Om)}
\newcommand{\Om}{\Omega}
\newcommand{\s}{\sigma}
\newcommand{\Bor}{\mathcal{B}}
\newcommand{\mbr}[1]{$(\Om_{#1}, \A_{#1})$}
\newcommand{\topspace}[1]{$(\Om_{#1}, \mathcal{O}_{#1})$}
\newcommand{\Fbar}{\bar{\mathcal{F}}(\Om, \A)}
\newcommand{\Fbarplus}{\bar{\mathcal{F}}_+(\Om, \A)}
\newcommand{\fmb}{\mathcal{F}(\Om, \A)}


% THEOREMSTYLES

\theorembodyfont{\upshape}
\theoremstyle{changebreak} % sorgt für Zeilenumbruch nach Titel und Nummer vor Name

\newtheorem{theorem}{Satz} % Sätze, sollte später evtl. noch Lemma und Korollar hinzufügen
\numberwithin{theorem}{chapter}

\newtheorem{mydef}[theorem]{Definition} % Definitionen
\numberwithin{theorem}{chapter}

\newtheorem{example}[theorem]{Beispiel} % Beispiele
\numberwithin{theorem}{chapter}

\newtheorem{proposition}[theorem]{Proposition} % Propositionen
\numberwithin{theorem}{chapter}

\newtheorem{remark}[theorem]{Bemerkung} % Bemerkungen
\numberwithin{theorem}{chapter}

\newtheorem{lemma}[theorem]{Lemma} % Lemmata
\numberwithin{theorem}{chapter}

\newtheorem*{proof*}{Beweis:}

\setlength\parindent{0pt} % kein Einzug bei neuem Paragraphen

% PDF-META
\hypersetup{pdftex,
            pdfauthor={Jonas Köppl, Maximilian Reif},
            pdftitle={Brückenkurs Mathematik - Inoffizielles Skript},
            pdfsubject={},
            pdfkeywords={},
            pdfproducer={},
            pdfcreator={},
            pdfpagemode=UseOutlines
}

\begin{document}

\begin{titlepage}
  \ \newline\newline\newline\newline\newline

  \begin{center}

  \huge Inoffizielles Skript\\
  \Huge\textbf{Brückenkurs Mathematik}\\
  \huge im Wintersemester 2018/2019\\
  \normalsize

  \vspace{1cm}
  \begin{tabular}[b]{l|l}
  \textbf{\ author}      & Jonas Köppl, Maximilian Reif, N.N.\\\
  \textbf{last change}   & \today\\\
  \textbf{github}        & \url{https://github.com/jonaskoeppl/brueckenkurs}
  \end{tabular}
  \vspace{1cm}

  \end{center}

\end{titlepage}

\tableofcontents


\chapter{Einführung in das mathematische Arbeiten}
% Subsection: 1.1. Ziel und Inhalt
\section{Ziele und Inhalt}
Im Vergleich mit vielen anderen Studiengängen, selbst mit den naturwissenschaftlichen,
hat das Mathematikstudium eine höhere Abbruchrate, und viele Studenten geben bereits während
oder nach dem ersten Semester auf. Ein Hauptgrund für diese hohe Abbruchquote ist, dass
sich die Art und Weise wie Mathematik an der Universität betrieben wird, grundlegend von dem
unterscheidet, was man aus der Schule gewohnt ist. Ziel des Brückenkurses ist es also die
Studierenden auf eine möglichst schonende Art auf ihr erstes Semester vorzubereiten und sie schoneinmal
mit der Hochschulmathematik vertraut zu machen. Zu den Inhalten zählen neben einer Einführung
in die Aussagenlogik auch die naive Mengenlehre und elementare Zahlentheorie. Diese bilden
eine gute Grundlage um anhand von Beispielen verschiedene Beweistechniken kennzulernen und
mit Übungsaufgaben zu vertiefen.

% Subsection 1.2. TODO: Schwierigkeiten zu Studienbeginn
\section{Schwierigkeiten beim Studienbeginn}
Auch aus eigener Erfahrung wissen wir, dass die folgenden beiden Punkte zu Problemen im ersten
Semester führen:
\begin{enumerate}
    \item
    \textbf{Die scheinbare Einfachheit des zu Beginn gelehrten Stoffes}. 
    Der in den Vorlesungen zu Beginn vorgetragene Stoff erscheint den meisten bereits aus der Schule bekannt 
    und leicht verständlich. Dies verführt dazu, sich auf seinem Schulwissen auszuruhen und den Zeitpunkt zu verpassen, 
    ab dem man sich spätestens richtig reinhängen sollte. Desweiteren sieht der Stoff zu Beginn nur leicht aus, doch es geht 
    auch nicht primär darum, was behandelt wird, sondern wie es behandelt wird. Dies dient dazu, elementare Beweistechniken
    anhand relativ einfacherer Beispiele kennenzulernen, und sich langsam an komplexere Sachverhalte heranzutasten. 
    \item
    \textbf{Der hohe Abstraktionsgrad}. Während die Mathematik in der Schule oft aus algorithmischem Lösen von standardisierten 
    Problemen bestand, ist an der Universität auch die Kreativität und Abstraktionsfähigkeit der Studierenden gefragt.
    Die wahre Entwicklung des Stoffes erfolgt nicht durch illustrative Beispiele, sondern innerhalb abstrakter Strukturen, die durch 
    möglichst wenig Attribute definiert werden, und zwischen denen nach und nach Querbeziehungen hergstellt werden. 
    Ein häufiger Fehler ist es, dass den Beweisen, in denen die Zusammenhänge aufgedeckt werden, nicht genug Bedeutung zugemessen wird.
    Dies führt zu einem mangelhaften Verständnis des Stoffes und somit später zu Frustration, da dieses Verständnis eine Grundvoraussetzung 
    für Vorlesungen in höheren Semestern ist. 
\end{enumerate}
Doch dies soll nicht zur Abschreckung dienen, sondern eher zur Motivation zur Bearbeitung der Übungsaufgaben und auch als kleine Warnung davor, 
sich in den ersten Wochen und Monaten auf den schulischen Lorbeeren auszuruhen. 

\section{Mathematisches Denken}

Unter mathematischem Denken verstehen wir eine kreative aber auch strukturierte Arbeitsweise, die insbesondere
auf das Lösen (selbst)gestellter Probleme abzielt. Die Lösungen dieser Probleme werden (nach mühseliger Arbeit) in Form 
von \textit{Beweisen} festgehalten, um sicherzustellen, dass auch andere Mathematiker die genutzten logischen Schlussfolgerungen und Schritte nachvollziehen können.
Oftmals ist der Beweis eines Satzes von größerer Bedeutung als das eigentliche Resultat. Dies ist beispielsweise der Fall, wenn man
für den Beweis eines Satzes eine neue Beweistechnik entwickelt hat, die sich auch in anderen Situationen gewinnbringend einsetzen lässt.
Wie auch die meisten anderen Wissenschaften lebt die moderne Mathematik von Kollaboration. Oftmals kommt man erst im Gespräch mit Anderen 
auf die richtige Idee um ein Problem zu lösen und auch das gegenseitige Erklären des aktuellen Stoffes trägt viel zum Verständnis bei.
\newline 
Um uns ein wenig aufzuwärmen, uns gegenseitig kennenzulernen  und ein wenig das logische Denken zu schulen gibt es nun auf den Übungsblättern ein paar einführende Rätsel. 

\chapter{Logische Grundlagen}

Bevor wir uns mit Beweisen beschäftigen können, müssen zunächst die dafür notwendigen logischen
Grundlagen erarbeitet werden. Hierzu benötigen wir eingangs ein paar Definitionen.

\begin{mydef}
Eine \textit{Aussage} ist ein sprachliches Gebilde, das entweder wahr oder falsch ist.
    Dabei ist es nicht erforderlich, sagen zu können, \textit{ob} die Aussage wahr oder falsch ist.
\end{mydef}  

Am besten veranschaulichen wir uns das mit einem Beispiel und einer Übungsaufgabe.

\begin{example}
    \begin{enumerate}[(i)]
        \item
        Es regnet. (Zeit- und ortsabhängig wahr oder falsch.)
        \item
        11 ist durch 5 teilbar.
        \item
        Es gibt unendlich viele Primzahlen. (Wahr, Beweis folgt.)
    \end{enumerate}
\end{example} 

Wir interessieren uns vor allem für mathematische Aussagen und deren Wahrheitsgehalt.
Zunächst benötigen wir jedoch noch Möglichkeiten, um mehrere Aussagen \textit{logisch zu verknüpfen}.

\begin{mydef}
    Wenn $P$ und $Q$ Aussagen sind, dann heißt $P \wedge Q$ die \textit{Konjunktion} von $P$ und $Q$.
    Der Wahrheitswert von $P \wedge Q$ ist definiert durch die Wahrheitswerte von $P$ und $Q$ mittels
    folgender \textit{Wahrheitstafel}: \newline
    \begin{tabular}{ c | c | c }
        $P$ & $Q$ & $P \wedge Q$ \\
        \hline
        w & w & w \\
        w & f & f \\
        f & w & f \\
        f & f & f \\
    \end{tabular}
    \newline 
    D.h. $P \wedge Q$ ist genau dann wahr, wenn $P$ und $Q$ beide wahr sind, und sonst falsch.
\end{mydef}

\begin{example}
    \begin{enumerate}[(i)]
        \item
        ($\sqrt{2}$ ist irrational) $\wedge$ ($\sqrt{2} > 0$) (wahr)
        \item
        $(2 + 2 = 4) \wedge (3 + 2 = 7)$ (falsch)
    \end{enumerate}
\end{example}

\begin{mydef}
    Wenn $P$ und $Q$ Aussagen sind, so heißt $P \vee Q$ die \textit{Disjunktion} von $P$ und $Q$.
    Die definierende Wahrheitstafel ist gegeben durch: \newline
    \begin{tabular}{ c | c | c }
        $P$ & $Q$ & $P \vee Q$ \\
        \hline
        w & w & w \\
        w & f & w \\
        f & w & w \\
        f & f & f \\
    \end{tabular}
    \newline 
    D.h. $P \vee Q$ ist genau dann wahr, wenn mindestens eine der beiden Teilaussagen wahr ist.
\end{mydef}

\begin{example}
    \begin{enumerate}[(i)]
        \item
        ($\sqrt{2}$ ist irrational) $\vee$ ($\sqrt{2} > 0$) (wahr)
        \item
        $(2 + 2 = 4) \vee (3 + 2 = 7)$ (wahr)
    \end{enumerate}
\end{example}

\begin{remark}
    Im Gegensatz zur Umgangssprache ist mit dem mathematischen oder stets das \textit{inklusive oder} gemeint.
    Möchte man das \textit{exklusive oder} verwenden, so nutzt man den Ausdruck \glqq entweder \ldots\ oder\ldots\grqq.
\end{remark}

\begin{mydef}
    Wenn $P$ eine Aussage ist, dann heißt $\neg P$ die \textit{Negation} von $P$.
    Definierende Wahrheitstafel: \newline
    \begin{tabular}{ c | c | c }
        $P$ &  $\neg P$ \\
        \hline
        w & f  \\
        f & w  \\
    \end{tabular}
\end{mydef}

\begin{mydef}
    Wenn $P$ und $Q$ zwei Aussagen sind, so heißt $P \Rightarrow Q$ (sprich: wenn $P$, dann $Q$) die \textit{Implikation} von Q durch P. 
    Definierende Wahrheitstafel: \newline
    \begin{tabular}{ c | c | c }
        $P$ & $Q$ & $P \Rightarrow Q$ \\
        \hline
        w & w & w \\
        w & f & f \\
        f & w & w \\
        f & f & w \\
    \end{tabular}
\end{mydef}

\begin{example}
    \begin{enumerate}[(i)]
        \item 
        Wenn $ 3 > 2$, dann teilt $5$ die Zahl $10$.  (wahr)
        \item 
        Wenn $2 > 3$, dann ist die Erde eine Scheibe. (wahr)
    \end{enumerate}
\end{example}

\begin{mydef}
    Wenn $P$ und $Q$ Aussagen sind, so heißt die Verknüpfung $(P \Rightarrow Q) \wedge (Q \Rightarrow P)$ die \textit{Äquivalenz} von $P$ und $Q$. 
    Abkürzend schreibt man auch $P \iff Q$. Die definierende Wahrheitstafel ist: \newline
    \begin{tabular}{ c | c | c }
        $P$ & $Q$ & $P \iff Q$ \\
        \hline
        w & w & w \\
        w & f & f \\
        f & w & f \\
        f & f & w \\
    \end{tabular}
\end{mydef}\ 
\newline 

Um all diese neuen Definitionen, und vor allem die Verwendung von Wahrheitstafeln, einzuüben, lohnt es sich, einige der zugehörigen Übungsaufgaben zu lösen. 
Ferner sollten Sie versuchen, sich mit selbstständig erstellten Beispielen besser mit dem Stoff vertraut zu machen. 
\chapter{Mengen und Elemente}

\section{Mengen und Teilmengen}
Im Folgenden geben wir eine kurze Einführung in die sogenannte \textit{naive Mengenlehre}. Eine Axiomatisierung der Mengenlehre ist möglich, wird aber hier nicht durchgeführt. 
Wer sich für eine Axiomatisierung interessiert, dem empfehlen wir die Vorlesung \glqq Mathematische Logik\grqq bei Prof. Dr. Kaiser, diese bietet sich aufgrund des hohen Abstraktionsgrades allerdings eher für spätere Semester an und ist deshalb inzwischen nur noch im Master anrechenbar. 

\begin{mydef}
    Eine \textit{Menge} $M$ ist eine Zusammenfassung von bestimmten wohlunterschiedenen Objekten unserer Anschauung oder unseres Denkens zu einem Ganzen. 
    Die in einer Menge zusammengefassten Objekte werden auch als \textit{Elemente} von $M$ bezeichnet. 
\end{mydef}

\begin{remark}
    Um besser über Mengen und ihre Elemente sprechen zu können, brauchen wir noch ein wenig Notation. 
    \begin{enumerate}[(i)]
        \item 
        Ist $x$ ein Element von $M$, so schreiben wir $x \in M$. 
        \item 
        Ist $x$ kein Element von $M$, so schreiben wir $x \notin M$. 
    \end{enumerate}
\end{remark}

\begin{example}
    In diesem Beispiel lernen wir einige Mengen und gängige Bezeichnungen kennen, damit wir uns später Schreibarbeit sparen können. 
    \begin{enumerate}[(i)]
        \item 
        Die Menge der \textit{natürlichen Zahlen}: $\N = \{1,2,3,...\}$.
        \item 
        Die Menge der \textit{natürlichen Zahlen} inklusive der $0$: $\N_0 = \{0,1,2,3,...\}$. 
        \item 
        $M = \{$ rot, gelb, blau $\}$. 
        \item 
        Die Menge aller Geraden und Dreiecke der Ebene
        \item 
        Die Menge aller Polynome $a_0 + a_1 x + ... + a_n x^n$ mit reellen Koeffizienten $a_0,a_1,...,a_n$. 
        \item 
        Die leere Menge $\emptyset$. Sie enthält keine Elemente. 
    \end{enumerate}
\end{example}

\begin{mydef}
    Seien $X,Y$ Mengen. Die Menge $X$ heißt \textit{Teilmenge} von $Y$ (und $Y$ heißt \textit{Obermenge} von $X$), 
    falls jedes Element von $X$ auch ein Element von $Y$ ist. In Zeichen: $X \subseteq Y$. Für die  Negation von $X \subseteq Y$ wird 
    $X \nsubseteq Y$ geschrieben. 
\end{mydef}

\begin{example}
    \begin{enumerate}[(i)]
        \item 
        Die Menge $\N$ ist eine Teilmenge von $\N_0$. 
        \item 
        Es gilt: $\N \subseteq \Z \subseteq \Q \subseteq \R$. 
    \end{enumerate}
\end{example}

\begin{mydef}
    Sei $X$ eine Menge, so bezeichne  $\mathcal{P}(X)$ die Menge aller Teilmengen von $X$. 
    $\mathcal{P}(X)$ heißt die \textit{Potenzmenge} von $X$. 
\end{mydef}

\begin{example}
    Betrachte die Menge $X = \{1,2\}$. Dann gilt: 
    \begin{align*}
        \mathcal{P}(X) = \{\emptyset, \{1\}, \{2\}, \{1,2\}\}
    \end{align*}
\end{example}


\section{Prädikate und Erzeugung von Teilmengen}

Betrachte zunächst die folgenden beiden Sätze: 
\begin{enumerate}[(i)]
    \item 
    Für jede natürliche Zahl $x$ gilt $x \geq 1$. 
    \item 
    $x$ ist kleiner als 5.
\end{enumerate}

Dann ist $(i)$ eine (wahre) Aussage, während man von $(ii)$ nicht sagen kann, ob es wahr oder falsch ist. Also ist $(ii)$ keine Aussage. 
Setzt man in $(ii)$ für $x$ eine beliebige natürliche Zahl ein, so erhält man aber eine Aussage. Im Satz $(i)$ kommt $x$ als \textit{gebundene} Variable vor, 
im Satz $(ii)$ als \textit{freie} bzw. \textit{ungebundene Variable}. 

\begin{mydef}
    Sätze (hier im umgangssprachlichen Sinn), in denen eine freie Variable $x$ vorkommt und die nach Ersetzen dieser Variablen $x$ durch ein mathematisches Objekt zu einer Aussage werden, 
    heißen \textit{Prädikate} (= Eigenschaften) von $x$. Sie werden bspw. mit $P(x)$ bezeichnet.
\end{mydef}

Einige oft verwendete Prädikate erhalten eigene Symbole. 

\begin{mydef}
    Sei $X$ eine Menge und $P(x)$ ein Prädikat.
    \begin{enumerate}
        \item 
        Die Aussage \glqq Für alle $x \in X$ gilt $P(x)$\grqq  wird abgekürzt durch: 
        \begin{align*}
            \forall x \in X : P(x).
        \end{align*}
        Das Symbol $\forall$ wird auch als \textit{Allquantor} bezeichnet. 
        \item 
        Die Aussage \glqq Es gibt ein $x \in X$, für das $P(x)$ gilt \grqq  wird abgekürzt durch: 
        \begin{align*}
            \exists x \in X : P(x).
        \end{align*}
        Das Symbol $\exists$ wird auch als \textit{Existenzquantor} bezeichnet. 
        \item 
        Die Aussage \glqq Es gibt genau ein $x \in X$, für das $P(x)$ gilt\grqq wird abgekürzt durch:
        \begin{align*}
            \exists ! x \in X : P(x). 
        \end{align*}
    \end{enumerate}
\end{mydef}

\begin{remark}
    Es gilt: 
    \begin{align*}
        [ \neg (\forall x \in X: P(x))] \iff [\exists x \in X: \neg P(x)]. \\\
        [ \neg (\exists x \in X: P(x))] \iff [\forall x \in X: \neg P(x)].
    \end{align*}
\end{remark}

\begin{mydef}
    Sei $X$ eine Menge und $P(x)$ ein Prädikat. Dann bezeichne 
    \begin{align*}
        \{ x \in X \mid \ P(x) \ \}
    \end{align*}
    diejenige Teilmenge von $X$, die aus allen Elementen von $X$ besteht, für die $P(x)$ wahr ist. Diese Darstellung einer Menge heißt auch \textit{intensional}. 
    Das bloße Aufzählen der beinhalteten Elemente wird als \textit{extensional} bezeichnet. 
\end{mydef}

\begin{example}
    \begin{enumerate}[(i)]
        \item 
        $\{n \in \N \mid \text{ n ist gerade } \} = \{2,4,6,...\}$. 
        \item 
        $\{n \in \Z \mid \text{ n ist ungerade } \} = \{...,-3,-1,1,3,...\}$. 
    \end{enumerate}   
    Die Beschreibung auf der linken Seite ist jeweils die intensionale Schreibweise, während auf der rechten Seite die extensionale Mengenschreibweise verwendet wird. 
\end{example}

Mit Hilfe der gerade eingeführten Notation können wir nun elementare Mengenoperationen definieren. 

\begin{mydef}
    Seien $X,Y$ zwei Mengen. 
    \begin{enumerate}[(i)]
        \item  
        $X \cup Y := \{x \mid x \in X \vee x \in Y\}$ heißt \textit{Vereinigung} von $X$ und $Y$. 
        \item 
        $X \cap Y := \{x \mid x \in X \wedge x \in Y\}$ heißt \textit{Durchschnitt} von $X$ und $Y$. 
        \item 
        $X \setminus Y := \{x \mid x \in X \wedge x \notin Y\}$ heißt \textit{relatives Komplement} von $Y$ in $X$. 
        \item 
        Ist $Y \subset X$, so heißt $X \setminus Y$ das \textit{Komplement} von $Y$ bzgl. $X$ und wird mit $Y^c$ bezeichnet. 
    \end{enumerate}
\end{mydef}

\begin{mydef}
    Durchschnitte und Vereinigungen von Mengen lassen sich auf beliebig viele Mengen verallgemeinern. 
    Sei dazu $I$ eine Menge und für alle $i \in I$ sei $A_i$ eine Menge. 
    \begin{enumerate}
        \item 
        Die Menge $\bigcup_{i \in I}A_i := \{x \mid \exists i \in I: x \in A_i\}$ heißt die Vereinigung der Mengen $A_i$, $i \in I$. 
        \item 
        Falls $I \neq \emptyset$ heißt die Menge $\bigcap_{i \in I}A_i := \{x \mid \forall i \in I: x \in A_i\}$  der Durchschnitt der Mengen $A_i$, $i \in I$.

    \end{enumerate} 
\end{mydef}

\begin{remark}
    \begin{enumerate}[(i)]
        \item 
        Schreibweise für $I = \N$: $\bigcap_{i=1}^{\infty}A_i$ bzw. $\bigcup_{i=1}^{\infty}A_i$.
        \item 
        Schreibweise für $I = \{1,...,n\}$: $\bigcap_{i=1}^n A_i$ bzw. $\bigcup_{i=1}^n A_i$.
    \end{enumerate}
\end{remark}
%TODO: Rechenregeln für Mengenoperationen, DeMorgan und Indexmengen
\include{chapter/04-Aufbau}
%TODO: Widerspruchsbeweis überarbeiten und Teilbarkeitsbeweise genauer anschauen.
\include{chapter/05-Relationen}
\end{document}
