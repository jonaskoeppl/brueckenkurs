\chapter{Einführung in das mathematische Arbeiten}
% Subsection: 1.1. Ziel und Inhalt
\section{Ziele und Inhalt}
Im Vergleich mit vielen anderen Studiengängen, selbst mit den naturwissenschaftlichen,
hat das Mathematikstudium eine höhere Abbruchrate, und viele Studenten geben bereits während
oder nach dem ersten Semester auf. Ein Hauptgrund für diese Begebenheit/Tatsache ist, dass
sich die Art und Weise wie Mathematik an der Universität betrieben wird, grundlegend von dem
unterscheidet, was man aus der Schule gewohnt ist. [...] Ziel des Brückenkurses ist es also die
Studierenden auf eine schonende/sanfte Art auf ihr erstes Semester vorzubereiten und sie schoneinmal
mit der Hochschulmathematik vertraut zu machen. Zu den Inhalten zählen neben einer Einführung
in die Aussagenlogik auch die naive Mengenlehre und elementare Zahlentheorie. Diese bilden
eine gute Grundlage um anhand von Beispielen verschiedene Beweistechniken kennzulernen und
mit Übungsaufgaben zu vertiefen.

% Subsection 1.2. TODO: Schwierigkeiten zu Studienbeginn
\section{Schwierigkeiten beim Studienbeginn}
Auch aus eigener Erfahrung wissen wir, dass die folgenden beiden Punkte zu Problemen im ersten
Semester führen:
\begin{enumerate}
    \item
    Die scheinbare Einfachheit des zu Beginn gelehrten Stoffes(schwer den Punkt zu finden an
    dem man anfangen sollte zu lernen, man lässt zu früh abreißen).
    \item
    Der hohe Abstraktionsgrad, mit Beispielen kommt man nicht mehr weit, man weiß nicht auf
    was man achten soll, abstrakte Strukturen schocken einen. (-> metrische Räume zur Einführung).
\end{enumerate}
Also: Nicht abreißen lassen, Übungsaufgaben machen, Fragen stellen, Lerngruppe suchen.

Zu Beginn / Zum Aufwärmen: Einführende Rätsel, bisschen warm werden, mit den Sitznachbarn sprechen.
