\chapter{Relationen und Abbildungen}

Ziel dieses Kapitel ist es, den Begriff der Abbildung (oder Funktion) mathematisch präzise zu fassen. 
Da eine Funktion ein Spezialfall einer Relation ist werden wir uns zuerst einmal mit Relationen im Allgemeinen befassen.

\section{Relationen}

\begin{mydef}
    Seien $A,B$ zwei nichtleere Mengen. Dann ist das \textit{kartesische Produkt} von $A$ und $B$ definiert durch:
    \begin{align*}
        A \times B := \{ (a,b) | a \in A \wedge b \in B \}. 
    \end{align*}
\end{mydef}

\begin{mydef}
    Seien $A,B$ zwei nichtleere Mengen und $A \times B$ das kartesische Produkt von $A$ und $B$. 
    Eine Teilmenge $R \subseteq A \times B$ heißt dann auch eine (zweistellige) \textit{Relation} (zwischen $X$ und $Y$).
    Gilt $X = Y$ so spricht man auch von einer Relation \textit{auf} $X$.  
    Anstatt von $(x,y) \in R$ schreibt man auch $xRy$. 
\end{mydef}

\begin{example}
    Die Folgenden Mengen sind Relationen: 
    \begin{enumerate}[(i)]
        \item 
         $R_1 := \{(a,b) | a \neq b \} \subseteq \Z \times \Z$.   
        \item 
        $R_2 := \{ (a,b) | a | b \} \subseteq \Z \times \Z$.
        \item 
        $R_3 := \{(a,b) | a \leq b\} \subseteq \R \times \R$. 
    \end{enumerate}
\end{example}

Eine Relation kann verschiedene Eigenschaften haben: 

\begin{mydef} 
    Seien $X$ eine nichtleere Meng und $R \subseteq X \times X$ eine Relation. $R$ heißt
    \begin{enumerate}[(i)]
        \item 
        \textit{reflexiv}, falls: $\forall x \in X: (x,x) \in R$,
        \item 
        \textit{symmetrisch}, falls: $\forall x,y \in X : (x,y) \in R \Rightarrow (y,x) \in R$,
        \item 
        \textit{antisymmetrisch}, falls: $\forall x,y \in X: (x,y) \in R \wedge (y,x) \in R \Rightarrow x=y$,
        \item 
        \textit{transitiv}, falls: $\forall x,y,z \in X: (x,y) \in R \wedge (y,z) \in R \Rightarrow (x,z) \in R$. 
    \end{enumerate}
\end{mydef}

TODO: Hierzu ein Beispiel. 

\begin{mydef}
    Eine Relation $R$ auf einer Menge $X$ wird als \textit{Äquivalenzrelation} bezeichnet, falls $R$ reflexiv, symmetrisch und transitiv ist. 
\end{mydef}

\begin{mydef}
    Eine Relation $R$ auf einer Menge $X$ wird als \textit{partielle Ordnung} bezeichnet, falls $R$ reflexiv, antisymmetrisch und transitiv ist. 
\end{mydef}

\begin{proposition}
    Sei $X$ eine Menge. Dann ist durch 
    \begin{align*}
        R := \{ (A,B) | A \subseteq B \}
    \end{align*}
    eine partielle Ordnung auf $\mathcal{P}(X)$ definiert. 
\end{proposition}

Beweis zur Übung? 

\begin{proposition}
    Sei $m \in \Z$. Dann ist durch 
    \begin{align*}
        R_m := \{(a,b)| m \mid (b-a)\}
    \end{align*}
    eine Äquivalenzrelation auf $\Z$ definiert. 
\end{proposition}

Beweis zur Übung? 

\section{Abbildungen}

\begin{mydef}
    Seien $A,B$  nichtleere Mengen und $R$ eine Relation zwischen $A$ und $B$. $R$ heißt 
    \begin{enumerate}[(i)]
        \item 
        \textit{linkstotal}, falls: $\forall a \in A \ \exists b \in B : (a,b) \in R$. 
        \item
        \textit{rechtseindeutig}, falls: $\forall a \in A \forall b,c \in B: (a,b) \in R \wedge (a,c) \in R \Rightarrow b = c $. 
    \end{enumerate}
    Eine linkstotale und rechtseindeutige Relation wird als \textit{Abbildung} oder auch als \textit{Funktion} bezeichnet. 
\end{mydef}

\begin{remark}
    Wenn man die Begriffe der Linktstotalität und der Rechtseindeutigkeit zusammenfasst erhält man 
    \begin{align}
        \forall a \in A \exists ! b \in B : (a,b) \in R. 
    \end{align}
    Dies wird in der Literatur vereinzelt auch als \textit{Funktionseigenschaft} bezeichnet. 
\end{remark}

\begin{remark}
    Funktionen können natürlich mit der für Relationen kennengelernten Notation aufgeschrieben werden, allerdings hat sich in der heutigen Zeit die folgende Notation
    für eine Funktion $f$ von $A$ nach $B$ durchgesetzt: 
    \begin{align*}
        f : A \to B, x \mapsto f(x)
    \end{align*}
    Hierbei ist es wichtig anzumerken, dass jede Funktionsdefinition aus zwei Bausteinen besteht: 
    \begin{enumerate}
        \item Angabe von Definitions- und Wertebereich,
        \item Angabe der Abbildungsvorschrift. 
    \end{enumerate}
\end{remark}

\begin{example}
    \begin{enumerate}[(i)]
        \item
        $f: \Z \to \Z, x \mapsto x^2 $. 
        \item 
        Sei $X$ eine nichtleere Menge. Dann heißt die Abbildung 
        \begin{align*}
            id_X : X \to X, x \mapsto x
        \end{align*}
        die \textit{Identität} auf $X$. 
        \item 
        Nur durch $x \mapsto x^2$ ist keine Funktion definiert, da Definitions- und Wertebereich nicht spezifiziert sind. 
    \end{enumerate}
\end{example}

Oftmals ist es notwendig Funktionen auf spezielle Eigenschaften zu untersuchen. Einige dieser Eigenschaften,
die euch im Laufe des Studiums noch häufiger begegnen werden, werden im Folgenden vorgestellt. %TODO: Satzbau. 

\begin{mydef}
    Seien $A,B$ nichtleere Mengen. Eine Abbildung $f:A \to B$ heißt
    \begin{enumerate}[(i)]
        \item
        \textit{injektiv}, falls: 
        \begin{align*}
            \forall a_1,a_2 \in A: f(a_1) = f(a_2) \Rightarrow a_1 = a_2, 
        \end{align*}
        \item 
        \textit{surjektiv}, falls: 
        \begin{align*}
            \forall b \in B \exists a \in A: f(a) = b,
        \end{align*}
        \item 
        \textit{bijektiv}, falls $f$ surjektiv und injektiv ist. 
    \end{enumerate}
\end{mydef}

\begin{example}
    Betrachte die Funktion 
    \begin{align*}
        f: \Z \to \Z, \ x \mapsto x^2.
    \end{align*}
    Dann ist $f$ nicht injektiv, denn es gilt: $f(-5) = 25 = f(5)$ und $-5 \neq 5$. Zudem ist $f$ auch nicht surjektiv, denn es gilt $f(x) = z^2 \geq 0$ für alle $x \in \Z$. 
\end{example}

\section{Bild und Urbild} 

\begin{mydef}
    Seien $X,Y$ nichtleere Mengen und sei $f:X \to Y$ eine Abbildung. 
    \begin{enumerate}[(i)]
        \item 
        Sei $A \subseteq X$. Dann heißt $f(A):=\{y \in Y | \exists x \in X : f(x) = y \}$ das \textit{Bild} von $A$ unter $f$. 
        \item 
        Sei $B \subseteq Y$. Dann heißt $f^{-1}(B):=\{x \in X | f(x) \in B \}$ das \textit{Urbild} von $B$ unter $f$. 
    \end{enumerate}
\end{mydef}

\begin{example}
    Betrachte die Abbildung 
    \begin{align*}
        f: \R \to \R, x \mapsto -x,
    \end{align*}
    Sowie die Mengen $A = [0,2]$ und $B = \{1,3,8,9\}$. Dann gilt: 
    \begin{align*}
        f(A) &= [-2,0], \\\
        f^{-1}(A) &= \{-1, -3,-8,-9\}. 
    \end{align*}
\end{example}

\begin{theorem}
    Seien $X,Y$ nichtleere Mengen, $f:X \to Y$ eine Abbildung, $I$ eine Menge, seien $A,B$ und $A_i$, $i \in I$, Teilmengen von $X$. Dann gilt: 
    \begin{enumerate}
        \item
        $f(A) = \emptyset \iff A = \emptyset$, 
        \item 
        $A \subseteq B \Rightarrow f(A) \subseteq f(B)$,
        \item 
        $ f(\bigcup_{i \in I}A_i) = \bigcup_{i \in I}f(A_i)$,
        \item 
        $ f(\bigcap_{i \in I}A_i) \subseteq \bigcap_{i \in I}f(A_i)$, wobei im Allgemeinen keine Gleichheit gilt. 
    \end{enumerate}
\end{theorem}

\begin{proof*}
    [Zur Übung]
\end{proof*}

\begin{theorem}
    Seien $X,Y$ nichtleere Mengen, $f:X \to Y$ eine Abbildung, $I$ eine Menge, seien $A,B$ und $A_i$, $i \in I$, Teilmengen von $Y$. Dann gilt: 
    \begin{enumerate}[(i)]
        \item 
        $f^{-1}(A) = \emptyset \iff A \cap f(X) = \emptyset$,
        \item 
        $A \subseteq B \Rightarrow f^{-1}(A) \subseteq f^{-1}(B)$,
        \item 
        $f^{-1}(\bigcup_{i \in I}A_i) = \bigcup_{i \in I}f^{-1}(A_i)$,
        \item 
        $f^{-1}(\bigcap_{i \in I}A_i) = \bigcap_{i \in I}f^{-1}(A_i)$,
        \item 
        $f^{-1}(Y \setminus A) = X \setminus f^{-1}(A)$.
    \end{enumerate}
\end{theorem}

\begin{proof*}
    [Zur Übung]
\end{proof*}

