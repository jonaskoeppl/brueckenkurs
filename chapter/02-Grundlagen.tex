\chapter{Logische Grundlagen}

Bevor wir uns mit Beweisen beschäftigen können, müssen zunächst die dafür
notwendigen logischen Grundlagen erarbeitet werden.
Hierzu benötigen wir eingangs ein paar Definitionen.

\begin{mydef}
Eine \textit{Aussage} ist ein sprachliches Gebilde, das entweder wahr oder
falsch ist.
Dabei ist es nicht erforderlich, sagen zu können, \textit{ob} die Aussage wahr
oder falsch ist.
\end{mydef}


Am besten veranschaulichen wir uns das mit einem Beispiel und einer Übungsaufgabe.


\begin{example}
\begin{enumerate}[(i)]
\item Es regnet. (Zeit- und ortsabhängig wahr oder falsch.)
\item 11 ist durch 5 teilbar. (Falsch.)
\item Es gibt unendlich viele Primzahlen. (Wahr, Beweis folgt.)
\item Es gibt unendlich viele Primzahlzwillinge (Unbewiesen, Stand 17.10.18.)
\end{enumerate}
\end{example}


Wir interessieren uns vor allem für mathematische Aussagen und deren
Wahrheitsgehalt.
Zunächst benötigen wir jedoch noch Möglichkeiten, um mehrere Aussagen
\textit{logisch zu verknüpfen}.


\begin{mydef}
Wenn $P$ und $Q$ Aussagen sind, dann heißt $P \wedge Q$ die
\textit{Konjunktion} von $P$ und $Q$.
Der Wahrheitswert von $P \wedge Q$ ist definiert durch die Wahrheitswerte von
$P$ und $Q$ mittels folgender \textit{Wahrheitstafel}:

\begin{table}[H]
\centering
\begin{tabular}{c|c|c}
$P$ & $Q$ & $P \wedge Q$ \\ \hline
w   & w   & w \\
w   & f   & f \\
f   & w   & f \\
f   & f   & f
\end{tabular}
\end{table}

Das heißt $P \wedge Q$ ist genau dann wahr, wenn $P$ und $Q$ beide wahr sind
und sonst falsch.
\end{mydef}


\begin{example}
\begin{enumerate}[(i)]
\item ($\sqrt{2}$ ist irrational) $\wedge$ ($\sqrt{2} > 0$) (Wahr.)
\item $(2 + 2 = 4) \wedge (3 + 2 = 7)$ (Falsch.)
\end{enumerate}
\end{example}


\begin{mydef}
Wenn $P$ und $Q$ Aussagen sind, so heißt $P \vee Q$ die \textit{Disjunktion}
von $P$ und $Q$.
Die definierende Wahrheitstafel ist gegeben durch:

\begin{table}[H]
\centering
\begin{tabular}{c|c|c}
$P$ & $Q$ & $P \vee Q$ \\ \hline
w   & w   & w \\
w   & f   & w \\
f   & w   & w \\
f   & f   & f
\end{tabular}
\end{table}

Das heißt $P \vee Q$ ist genau dann wahr, wenn mindestens eine der beiden
Teilaussagen wahr ist.
\end{mydef}


\begin{example}
\begin{enumerate}[(i)]
\item ($\sqrt{2}$ ist irrational) $\vee$ ($\sqrt{2} > 0$) (Wahr.)
\item $(2 + 2 = 4) \vee (3 + 2 = 7)$ (Wahr.)
\end{enumerate}
\end{example}


\begin{remark}
Im Gegensatz zur Umgangssprache ist mit dem mathematischen oder stets das
\textit{inklusive oder} gemeint.
Möchte man das \textit{exklusive oder} verwenden, so nutzt man den Ausdruck
\glqq entweder \ldots\ oder\ldots\grqq.
\end{remark}


\begin{mydef}
Wenn $P$ eine Aussage ist, dann heißt $\neg P$ die \textit{Negation} von $P$.
Definierende Wahrheitstafel:

\begin{table}[H]
\centering
\begin{tabular}{c|c}
$P$ & $\neg P$ \\ \hline
w   & f \\
f   & w
\end{tabular}
\end{table}
\end{mydef}


\begin{mydef}
Wenn $P$ und $Q$ zwei Aussagen sind, so heißt $P \Rightarrow Q$
(sprich: wenn $P$, dann $Q$) die \textit{Implikation} von Q durch P.
Definierende Wahrheitstafel:

\begin{table}[H]
\centering
\begin{tabular}{c|c|c}
$P$ & $Q$ & $P \Rightarrow Q$ \\ \hline
w   & w   & w \\
w   & f   & f \\
f   & w   & w \\
f   & f   & w
\end{tabular}
\end{table}
\end{mydef}


\begin{example}
\begin{enumerate}[(i)]
\item Wenn $ 3 > 2$, dann teilt $5$ die Zahl $10$. (Wahr.)
\item Wenn $2 > 3$, dann ist die Erde eine Scheibe. (Wahr.)
\end{enumerate}
\end{example}


% TODO: Regen => Straße nass


\begin{mydef}
Wenn $P$ und $Q$ Aussagen sind, so heißt die Verknüpfung
$(P \Rightarrow Q) \wedge (Q \Rightarrow P)$ die \textit{Äquivalenz} von
$P$ und $Q$.
Abkürzend schreibt man auch $P \iff Q$.
Die definierende Wahrheitstafel ist:

\begin{table}[H]
\centering
\begin{tabular}{c|c|c}
$P$ & $Q$ & $P \iff Q$ \\ \hline
w   & w   & w \\
w   & f   & f \\
f   & w   & f \\
f   & f   & w
\end{tabular}
\end{table}
\end{mydef}

\vspace*{3em}
Um all diese neuen Definitionen, und vor allem die Verwendung von
Wahrheitstafeln, einzuüben, lohnt es sich, einige der zugehörigen
Übungsaufgaben zu lösen.
Ferner sollten Sie versuchen, sich mit selbstständig erstellten Beispielen
besser mit dem Stoff vertraut zu machen.
