\chapter{Einführung in das mathematische Arbeiten}
% Subsection: 1.1. Ziel und Inhalt
\section{Ziele und Inhalt}
Im Vergleich mit vielen anderen Studiengängen, selbst mit den naturwissenschaftlichen,
hat das Mathematikstudium eine höhere Abbruchrate, und viele Studenten geben bereits während
oder nach dem ersten Semester auf. Ein Hauptgrund für diese hohe Abbruchquote ist, dass
sich die Art und Weise wie Mathematik an der Universität betrieben wird, grundlegend von dem
unterscheidet, was man aus der Schule gewohnt ist. Ziel des Brückenkurses ist es also die
Studierenden auf eine möglichst schonende Art auf ihr erstes Semester vorzubereiten und sie schoneinmal
mit der Hochschulmathematik vertraut zu machen. Zu den Inhalten zählen neben einer Einführung
in die Aussagenlogik auch die naive Mengenlehre und elementare Zahlentheorie. Diese bilden
eine gute Grundlage um anhand von Beispielen verschiedene Beweistechniken kennzulernen und
mit Übungsaufgaben zu vertiefen.

% Subsection 1.2. TODO: Schwierigkeiten zu Studienbeginn
\section{Schwierigkeiten beim Studienbeginn}
Auch aus eigener Erfahrung wissen wir, dass die folgenden beiden Punkte zu Problemen im ersten
Semester führen:
\begin{enumerate}
    \item
    \textbf{Die scheinbare Einfachheit des zu Beginn gelehrten Stoffes}. 
    Der in den Vorlesungen zu Beginn vorgetragene Stoff erscheint den meisten bereits aus der Schule bekannt 
    und leicht verständlich. Dies verführt dazu, sich auf seinem Schulwissen auszuruhen und den Zeitpunkt zu verpassen, 
    ab dem man sich spätestens richtig reinhängen sollte. Desweiteren sieht der Stoff zu Beginn nur leicht aus, doch es geht 
    auch nicht primär darum, was behandelt wird, sondern wie es behandelt wird. Dies dient dazu, elementare Beweistechniken
    anhand relativ einfacherer Beispiele kennenzulernen, und sich langsam an komplexere Sachverhalte heranzutasten. 
    \item
    \textbf{Der hohe Abstraktionsgrad}. Während die Mathematik in der Schule oft aus algorithmischem Lösen von standardisierten 
    Problemen bestand, ist an der Universität auch die Kreativität und Abstraktionsfähigkeit der Studierenden gefragt.
    Die wahre Entwicklung des Stoffes erfolgt nicht durch illustrative Beispiele, sondern innerhalb abstrakter Strukturen, die durch 
    möglichst wenig Attribute definiert werden, und zwischen denen nach und nach Querbeziehungen hergstellt werden. 
    Ein häufiger Fehler ist es, dass den Beweisen, in denen die Zusammenhänge aufgedeckt werden, nicht genug Bedeutung zugemessen wird.
    Dies führt zu einem mangelhaften Verständnis des Stoffes und somit später zu Frustration, da dieses Verständnis eine Grundvoraussetzung 
    für Vorlesungen in höheren Semestern ist. 
\end{enumerate}
Doch dies soll nicht zur Abschreckung dienen, sondern eher zur Motivation zur Bearbeitung der Übungsaufgaben und auch als kleine Warnung davor, 
sich in den ersten Wochen und Monaten auf den schulischen Lorbeeren auszuruhen. 

\section{Mathematisches Denken}

Unter mathematischem Denken verstehen wir eine kreative aber auch strukturierte Arbeitsweise, die insbesondere
auf das Lösen (selbst)gestellter Probleme abzielt. Die Lösungen dieser Probleme werden (nach mühseliger Arbeit) in Form 
von \textit{Beweisen} festgehalten, um sicherzustellen, dass auch andere Mathematiker die genutzten logischen Schlussfolgerungen und Schritte nachvollziehen können.
Oftmals ist der Beweis eines Satzes von größerer Bedeutung als das eigentliche Resultat. Dies ist beispielsweise der Fall, wenn man
für den Beweis eines Satzes eine neue Beweistechnik entwickelt hat, die sich auch in anderen Situationen gewinnbringend einsetzen lässt.
Wie auch die meisten anderen Wissenschaften lebt die moderne Mathematik von Kollaboration. Oftmals kommt man erst im Gespräch mit Anderen 
auf die richtige Idee um ein Problem zu lösen und auch das gegenseitige Erklären des aktuellen Stoffes trägt viel zum Verständnis bei.
\newline 
Um uns ein wenig aufzuwärmen, uns gegenseitig kennenzulernen  und ein wenig das logische Denken zu schulen gibt es nun auf den Übungsblättern ein paar einführende Rätsel. 
