\chapter{Mengen und Elemente}

\section{Mengen und Teilmengen}
Im Folgenden geben wir eine kurze Einführung in die sogenannte \textit{naive Mengenlehre}. Eine Axiomatisierung der Mengenlehre ist möglich, wird aber hier nicht durchgeführt. 
Wer sich für eine Axiomatisierung interessiert, dem empfehlen wir die Vorlesung \glqq Mathematische Logik \grqq bei Prof. Dr. Kaiser, diese bietet sich aufgrund des hohen Abstraktionsgrades allerdings eher für spätere Semester an. 

\begin{mydef}
    Eine \textit{Menge} $M$ ist eine Zusammenfassung von bestimmten wohlunterschiedenen Objekten unserer Anschauung oder unseres Denkens zu einem Ganzen. 
    Die in einer Menge zusammengefassten Objekte werden auch als \textit{Elemente} von $M$ bezeichnet. 
\end{mydef}

\begin{remark}
    Um besser über Mengen und ihre Elemente sprechen zu können brauchen wir noch ein wenig Notation. 
    \begin{enumerate}[(i)]
        \item 
        Ist $x$ ein Element von $M$, so schreiben wir $x \in M$. 
        \item 
        Ist $x$ kein Element von $M$, so schreiben wir $x \notin M$. 
    \end{enumerate}
\end{remark}

\begin{example}
    In diesem Beispiel lernen wir einige Mengen und gängige Bezeichnungen kennen, damit wir uns später Schreibarbeit sparen können. 
    \begin{enumerate}[(i)]
        \item 
        Die Menge der \textit{natürlichen Zahlen} : $\N = \{1,2,3,...\}$.
        \item 
        Die Menge der \textit{natürlichen Zahlen} inklusive der $0$: $N_0 = \{0,1,2,3,...\}$. 
        \item 
        $M = \{$ rot, gelb, blau $\}$. 
        \item 
        Die Menge aller Geraden und Dreiecke der Ebene
        \item 
        Die Menge aller Polynome $a_0 + a_1 x + ... + a_n x^n$ mit reellen Koeffizienten $a_0,a_1,...,a_n$. 
        \item 
        Die leere Menge $\emptyset$. Sie enthält keine Elemente. 
    \end{enumerate}
\end{example}

\begin{mydef}
    Seien $X,Y$ Mengen. Die Menge $X$ heißt \textit{Teilmenge} von $Y$ (und $Y$ heißt \textit{Obermenge} von $X$), 
    falls jedes Element von $X$ auch ein Element von $Y$ ist. In Zeichen: $X \subseteq Y$. Für die  Negation von $X \subseteq Y$ wird 
    $X \nsubseteq Y$ geschrieben. 
\end{mydef}

\begin{example}
    \begin{enumerate}[(i)]
        \item 
        Die Menge $\N$ ist eine Teilmenge von $\N_0$. 
        \item 
        Es gilt: $\N \subseteq \Z \subseteq \Q \subseteq \R$. 
    \end{enumerate}
\end{example}

\begin{mydef}
    Sei $X$ eine Menge, so bezeichne  $\mathcal{P}(X)$ die Menge aller Teilmengen von $X$. 
    $\mathcal{P}(X)$ heißt die \textit{Potenzmenge} von $X$. 
\end{mydef}

\begin{example}
    Betrachte die Menge $X = \{1,2\}$. Dann gilt: 
    \begin{align*}
        \mathcal{P}(X) = \{\emptyset, \{1\}, \{2\}, \{1,2\}\}
    \end{align*}
\end{example}


\section{Prädikate und Erzeugung von Teilmengen}

Betrachte zunächst die folgenden beiden Sätze: 
\begin{enumerate}[(i)]
    \item 
    Für jede natürliche Zahl $x$ gilt $x \geq 1$. 
    \item 
    $x$ ist kleiner als 5.
\end{enumerate}

Dann ist $(i)$ eine (wahre Aussage), während man von $(ii)$ nicht sagen kann, ob es wahr oder falsch ist, also ist $(ii)$ keine Aussage. 
Setzt man in $(ii)$ für $x$ eine beliebige natürliche Zahl ein, so erhält man aber eine Aussage. Im Satz $(i)$ kommt $x$ als \textit{gebundene} Variable vor, 
im Satz $(ii)$ als \textit{freie} bzw. \textit{ungebundene Variable}. 

\begin{mydef}
    Sätze (hier im umgangssprachlichen Sinn), in denen eine freie Variable $x$ vorkommt und die nach Ersetzen dieser Variablen $x$ durch ein mathematisches Objekt zu einer Aussage werden, 
    heißen \textit{Prädikate} (= Eigenschaften) von $x$. Sie werden bspw. mit $P(x)$ bezeichnet.
\end{mydef}

Einige oft verwendete Prädikate erhalten eigene Symbole. 

\begin{mydef}
    Sei $X$ eine Menge und $P(x)$ ein Prädikat.
    \begin{enumerate}
        \item 
        Die Aussage \glqq Für alle $x \in X$ gilt $P(x)$ \grqq  wird abgekürzt durch: 
        \begin{align*}
            \forall x \in X : P(x).
        \end{align*}
        Das Symbol $\forall$ wird auch als \textit{Allquantor} bezeichnet. 
        \item 
        Die Aussage \glqq Es gibt ein $x \in X$, für das $P(x)$ gilt \grqq  wird abgekürzt durch: 
        \begin{align*}
            \exists x \in X : P(x).
        \end{align*}
        Das Symbol $\exists$ wird auch als \textit{Existenzquantor} bezeichnet. 
        \item 
        Die Aussage \glqq Es gibt genau ein $x \in X$, für das $P(x)$ gilt \grqq wird abgekürzt durch:
        \begin{align*}
            \exists ! x \in X : P(x). 
        \end{align*}
    \end{enumerate}
\end{mydef}

\begin{remark}
    Es gilt: 
    \begin{align*}
        [ \neg (\forall x \in X: P(x))] \iff [\exists x \in X: \neg P(x)]. \\\
        [ \neg (\exists x \in X: P(x))] \iff [\forall x \in X: \neg P(x)].
    \end{align*}
\end{remark}

\begin{mydef}
    Sei $X$ eine Menge und $P(x)$ ein Prädikat. Dann bezeichne 
    \begin{align*}
        \{ x \in X \mid \ P(x) \ \}
    \end{align*}
    diejenige Teilmenge von $X$, die aus allen Elementen von $X$ besteht, für die $P(x)$ wahr ist. Diese Darstellung einer Menge heißt auch \textit{intensional}. 
    Das bloße Aufzählen der beinhalteten Elemente wird als \textit{extensional} bezeichnet. 
\end{mydef}

\begin{example}
    \begin{enumerate}[(i)]
        \item 
        $\{n \in \N \mid \text{ n ist gerade } \} = \{2,4,6,...\}$. 
        \item 
        $\{n \in \Z \mid \text{ n ist ungerade } \} = \{...,-3,-1,1,3,...\}$. 
    \end{enumerate}   
    Die Beschreibung auf der linken Seite ist jeweils die intensionale Schreibweise, während auf der rechten Seite die extensionale Mengenschreibweise verwendet wird. 
\end{example}\ 
Mit Hilfe der gerade eingeführten Notation können wir nun elementare Mengenoperationen definieren. 

\begin{mydef}
    Seien $X,Y$ zwei Mengen. 
    \begin{enumerate}[(i)]
        \item  
        $X \cup Y := \{x \mid x \in X \vee x \in Y\}$ heißt \textit{Vereinigung} von $X$ und $Y$. 
        \item 
        $X \cap Y := \{x \mid x \in X \wedge x \in Y\}$ heißt \textit{Durchschnitt} von $X$ und $Y$. 
        \item 
        $X \setminus Y := \{x \mid x \in X \wedge x \notin Y\}$ heißt \textit{relatives Komplement} von $Y$ in $X$. 
        \item 
        Ist $Y \subset X$, so heißt $X \setminus Y$ das \textit{Komplement} von $Y$ bzgl. $X$ und wird mit $Y^c$ bezeichnet. 
    \end{enumerate}
\end{mydef}

\begin{mydef}
    Durchschnitte und Vereinigungen von Mengen lassen sich auf beliebig viele Mengen verallgemeinern. 
    Sei dazu $I$ eine Menge und für alle $i \in I$ sei $A_i$ eine Menge. 
    \begin{enumerate}
        \item 
        Die Menge $\bigcup_{i \in I}A_i := \{x \mid \exists i \in I: x \in A_i\}$ heißt die Vereinigung der Mengen $A_i$, $i \in I$. 
        \item 
        Falls $I \neq \emptyset$ heißt die Menge $\bigcap_{i \in I}A_i := \{x \mid \forall i \in I: x \in A_i\}$  der Durchschnitt der Mengen $A_i$, $i \in I$.

    \end{enumerate} 
\end{mydef}

\begin{remark}
    \begin{enumerate}[(i)]
        \item 
        Schreibweise für $I = \N$: $\bigcap_{i=1}^{\infty}A_i$ bzw. $\bigcup_{i=1}^{\infty}A_i$.
        \item 
        Schreibweise für $I = \{1,...,n\}$: $\bigcap_{i=1}^n A_i$ bzw. $\bigcup_{i=1}^n A_i$.
    \end{enumerate}
\end{remark}